\documentclass[10pt,a4paper,final]{article}
\usepackage[utf8]{inputenc}
\usepackage[english]{babel}
\usepackage{amsmath}
\numberwithin{equation}{section}
\usepackage[font=small,labelfont=bf]{caption}
\newenvironment{Table}
  {\par\bigskip\noindent\minipage{\columnwidth}\centering}
    {\endminipage\par\bigskip}
\usepackage{amsfonts}
\usepackage{amssymb}
\usepackage{makeidx}
\usepackage{graphicx}
\usepackage{lmodern}
\usepackage{hyperref}
\usepackage[left=0.5in,right=0.5in,top=1in,bottom=1in,
						headheight=15pt]{geometry}
\usepackage{blindtext}
\usepackage[bottom]{footmisc}
\usepackage[square, comma, numbers]{natbib}
\usepackage{caption}
\usepackage{subcaption}
\usepackage{fancyvrb}
\usepackage{fancyhdr}
\pagestyle{fancy}
\fancyhead{}
\fancyfoot{}
\fancyhead[L]{Physics 565}
\fancyhead[C]{- - - Final Project Writeup - - -}
\fancyhead[R]{Lentner, G.}
\fancyfoot[C]{\large \thepage}
\usepackage{multicol}
\renewcommand{\headrulewidth}{0pt}
\renewcommand{\footrulewidth}{0pt}
\renewcommand{\bibsection}{}

\begin{document}
	
	\thispagestyle{plain}

	\begin{flushleft}
		\LARGE{\textbf{A stochastic algorithm for quantifying partial solutions
		of the Drake Equation}}
	\end{flushleft}
	
	\noindent\makebox[\linewidth]{\rule{\textwidth}{0.75pt}}

	\bigskip
		
	\begin{flushright}
	\begin{minipage}[t]{0.85\textwidth}	
		\par{\Large{\textbf{G. Lentner}}\par}
		\bigskip
		\par{\small{\textit{
				LL15 Natural Science Building, University of Louisville,
				Louisville, KY 40292 USA}}\par}
		\par{\small{\textit{
				email: geoffrey.lentner@louisville.edu}}\par}
	\end{minipage}
	\end{flushright}

	\vspace{\baselineskip}

	\begin{flushright}
	\begin{minipage}[t]{0.85\textwidth}
		\textbf{Abstract: } Here I review a new code developed to analyze 
		the quantitative effects of estimates on statistical profiles for the distribution
		of planets in galaxies as well 
		as the number of planets distributed therein. This application (hereafter referred 
		to as GAIA) models galaxies by interpolating $N$ (number of) pseudo-random 
		positions from arbitrary probability density functions in three dimensions given
		to it. GAIA takes these positions and does a \textit{nearest neighbor} analysis
		to estimate the expected separations between bodies and solves for the sample 
		standard deviation to establish a confidence interval for its results. 
		Conceptually, the idea here is that we now have an understanding of how 
		the early terms in the Drake equation are distributed spatially throughout the galaxy.
		If the trailing terms in the Drake equation (such as whether a civilization 
		survives itself) cannot be modeled with a spacial profile because it is a function 
		of local evolution and stochastic events in time that may continue to elude refinement; 
		we can assume all possible end results, $N$, and 
		distribute planets based on the profiles we know and study how the galaxy might look
		under these conditions.
		GAIA is a tool that, if used properly, can have implications for the Fermi Paradox,
		shedding light on how preferable a location we live in for detecting our neighbors.
		In this paper, I'll outline the code's construction and the user interface, as well
		as describe it's current issues and limitations. The code is available from its
		master online repository: \texttt{\small http://github.com/gLENTNER/ProjectGAIA.git}
	\end{minipage}
	\end{flushright}

	\begin{flushright}
	\begin{minipage}[t]{0.85\textwidth}
		\par{\small{\textbf{Key words: } statistical Drake Equation, Monte Carlo, 
		Fermi Paradox, SETI, probability densities}\par}
	\end{minipage}
	\end{flushright}

	\vspace*{2\baselineskip}

	\begin{multicols}{2}[]
		
		\section{Introduction}
			
			The prospect of detecting (or making contact with) an Extraterrestrial
			Intelligence (ETI) is supreme and its implications for our collective
			human perspective goes without saying. There are many ways in which an
			estimate on the total number of ETIs in the Milky Way can be expressed,
			the most prominent construction being the Drake Equation devised in 1961 by
			Frank Drake at the first meeting of SETI \cite{drake}. It was initially only
			intended as a thought tool to prompt intellectual curiosity; however, it is
			often used today as a serious method for constructing estimations. It has
			been revised \cite{sagan} to be more explicit in its parameterization and
			revisited often. In my present discussion, I'm going to adopt the below
			formulation \cite{steven}, which represents a typical construction considered
			today:

			\begin{equation}
				N = R \cdot f_p \cdot n_e \cdot f_L \cdot f_i \cdot f_c \cdot
				H_T \cdot H_*
			\end{equation}

			\noindent
			where the parameters are defined in Table \ref{Table:params}. The particular
			parameterization is not really what is important, but more so that we include
			all the spatially distributed qualities necessary for the later terms. 
			I'll explain. I think it is the case that the components concerning evolution
			of life and intelligence are localized; that is, whether it occurs is
			
			\begin{Table}
			\captionof{table}{Table of Drake Equation Parameters.}
			\begin{tabular}{ll} \hline\hline
				& \\
				$R$   & Average star production rate.\\
				$f_g$ & Fraction of stars that are single F, G, or K dwarfs.\\
				$f_p$ & Fraction of stars with planets.\\
				$n_e$ & Number of suitable planets per star.\\
				$f_L$ & Fraction of suitable planets which evolve life.\\
				$f_i$ & Fraction of life bearing planets which develop\\
					& intelligent life.\\
				$f_c$ & Fraction of planets with intelligent life which\\
					& develop a technological civilization.\\
				$H_c$ & Characteristic time for evolution of a civilization.\\
				$H_*$ & Characteristic decay time for galactic star\\
					& formation rate.
			\end{tabular}
			\label{Table:params}
			\end{Table}
			
			\noindent
			dependent on scales of the solar system, not the galaxy. Furthermore, they are more
			difficult (relative to other terms in the equation) to model as of today because
			we simply don't have enough data. A component often included in current studies
			of this kind is a factor that deals with the potential for extinction events or
			destruction of the biosphere (potentially to later regenerate or otherwise lost
			forever). This could be potentially handled though by considering what is now
			being referred to as the Galactic Habitable Zone.
			
			To restate it briefly: I am proposing for consideration a new treatment that simply
			models the distribution of relative probabilities of characteristics necessary for
			life to evolve and survive. We can posit a total number based on current best estimates
			of the discriminating factors for ETIs and then build Monte Carlo Realizations (MCRs) of
			the galaxy to study the impact of such a result. There is the potential here as well
			to take a backward look  and place constraints on $N$. If we have a defined level of 
			confidence in our statistical models for habitability in the galaxy, we can consider 
			our confidence
			in the lack of ETIs within a local volume as a discriminant for a range of solutions that
			would predict a neighbor within that volume to the same level of confidence.

			To this end, I've developed a new code (here after referred to as GAIA) that does just
			this. Given a set of probability density functions (PDFs) in any of three dimensions,
			it builds populations of a requested size that meet those statistics and performs
			a nearest neighbor analysis. The output is the expected separation between neighbors
			as a function of galactocentric radius and the sample standard deviation of that measure
			based on the number of trials (MCRs) constructed.

			In the next section I'll give a brief overview of the types of profiles that might yield
			the best results; however, this is more so a description of the tool, and not my employment
			of it (which is forthcoming).

		\section{Probability and Statistics}
	
			This paper is not concerned with a lengthy discussion of probability theory
			and techniques in mixing PDFs and their resultant cumulative distribution
			functions (CDFs). With that said, I'll include a brief discussion of some
			of types of profiles I think are appropriate to include and how they are 
			treated.

			In the radial dimension, we can model the stellar number density with a mass
			density profile. One of the first requisites is that a planet be in orbit
			around a star, so we must go where the stars are. The surface density profile
			of the galaxy is generally speaking something of a decaying exponential.

			\begin{equation}
				\rho(r) \approx n_0 \, \text{exp}\left( -r / R_D \right)
			\end{equation}
		
			Additionally, the vertical structure of the galaxy can be modeled with a decaying
			exponential. This gives us planets distributed throughout the galaxy consistent
			with the stellar population. As for an angular profile, the spiral arms contain
			the denser star forming regions, and one might think to include something of this
			nature; however, F, G, and K type stars are not so bound to the spirals as they
			are much longer lived. This means that they are largely distributed isotropically
			in the disk.

			We might also consider the metallicity gradient of the disk as terrestrial planets
			are relatively speaking more likely to be found around stars born of metal rich
			material. A radial profile that respects this is warranted.

			Lastly, as I stated previously, there are regions where it is reasonable to think
			it is more/less likely that a biosphere would remain undisturbed in comparison
			to other regions. If a system is in a more dense region of the galaxy, it is more
			likely to have material thrown in to it's inner solar system. As well, supernova
			events are more of a threat. This has implications for life in the spiral arms.
			A system is at it's best chance (I'll hypothesize) when it is not passing in and
			out of the arms. So we might expect that there is a profile consisting of a tight
			gaussian-like curve around the radius of co-rotation. This is the radius at which
			the system would have a circular velocity equal to that of the pattern speed of
			the spiral arms. This would create a belt of habitability in the galaxy.

			These types of considerations can be included in a set of profiles to be mixed,
			integrated, and used by GAIA to analyze what the implications would be for the
			expected neighbor separations between ETIs in that environment.
			
							
		\section{Code Structure}
				
				GAIA is programmed with an object oriented design patter. There is a 
				hierarchical structure of ownership and management. In the following
				sections, I'll layout the components of the code and brief descriptions
				of their responsibilities. The structure is depicted graphically
				in Figure \ref{fig:structure} at the top of the next page.

			\begin{figure*}[t!]
				\captionsetup{width=0.8\textwidth}
				\centering
				\includegraphics[width=\textwidth]{../figures/GAIAstructure.pdf}
				\caption{Above is a schematic diagram that shows how the GAIA 
				application is constructed. In general, the scope flows down
				from left to right (with the exception of the parser, who is called
				by all the objects, but simply instantiated first by the application
				in GAIAsolution).}
				\label{fig:structure}
			\end{figure*}

			\subsection{Tools}
			
				Throughout the simulation, there are two operations that happen with high
				frequency. That is the generation of sets of psuedorandom numbers and
				the act of resampling data onto a fixed grid. These algorithms already
				exist in the public domain in abundance. My goal was not to ``reinvent
				the wheel'' so to say. 
				
				In the case of the PRNG, I needed something very
				high performance with a long period. Most all of the scripting languages
				employ the exact same algorithm (mt19937). In my case, I not only wanted
				this for C++, but I needed it implemented as a \textit{class} so I could
				spawn multiple independent threads for generating numbers in parallel.
				The new C++11 standard in fact has such an implementation, 
				\texttt{std::mt19937\_64}. As it turns out, my code produces identical
				output as the C++11 implementation. After I discovered it's existence
				I decided to just stick with mine because it was local and didn't require
				the use of experimental libraries.

				As for the interpolation, I wanted an algorithm that was \textit{in house},
				fast, parallelized, and specialized for my purposes. I needed to know 
				exactly what it was doing, and I needed it to do some extra house-keeping
				for me.

			\subsubsection{Pseudo-Random Numbers: mt19937}
				
				GAIA uses the Mersenne Twister as developed by Matsumoto and
				Nishimuro \cite{mt1,mt2}. The algorithm is a generalized feedback
				shift register. The code uses the 64-bit version of mt19937 as
				written by the developers themselves. The C-code was used to
				construct a C++ class and stripped to its essentials. As I stated
				previously, the object was necessary so that I could construct an
				array of independent instances (seeded separately). With this,
				I developed a routine (part of the \texttt{PopulationManager})
				that constructs an array of pseudorandom numbers by have multiple
				generators working on delegated segments of the array in parallel.

			\subsubsection{SPLine INTERpolation}

				During the initialization process, the PDFs can optionally be
				read in as empirical data from an ascii file. If this is used,
				it is necessary to fix that data onto the working line-space 
				for that dimension. This is for two reasons. First, there may
				be more profiles along that dimension, in which case it will be
				necessary to have the data at the same positions in order to
				properly combine them. Second, not only do we want to ensure
				the resolution of the data for the integration process, but
				GAIA uses midpoints a la Simpson's 3/8th's rule. Also, and more 
				important, for every trial MCR that the code builds,
				we need to be able to compare the same radial values when averaging.
				
				GAIA uses natural cubic spline interpolation to resample data. The
				code is my own, the algorithm is not. I simply referenced the nicely
				developed explanation available via \textit{Wikipedia} 
				\cite{splinter, thomas}.

				I wanted to have an object that constructs the spline polynomials
				upon being instantiated and retains these data. For obvious reasons,
				I didn't want GAIA to be building the polynomials every time it 
				needed to interpolate. Furthermore, many implementations do not
				incorporate a sorting algorithm. Generally speaking this is not
				necessary because you would like to be able to interpolate around
				some function that might not be single-valued. As far as we are 
				concerned however, we want our data in ascending order. The
				\texttt{splinter} object uses an adaptation of the \textit{quicksort}
				algorithm. Quicksort is a recursive algorithm that partitions
				an array of values around a pivot value and then repeats on either
				side of the pivot. The largest of the arrays feed to \texttt{splinter}
				are the random numbers (which are uniformly distributed), so the pivot
				was taken as the midpoint, as there is no gain by trying to make
				guesses at an optimal pivot. The algorithm was adapted to work on
				two dimensional arrays (using the column as the sorting array) and
				the other columns strung along with it. Further, the function
				was overloaded with a wrapper that segments the array into intervals.
				The array is pre-sorted in parallel by having the recursive function
				operate on independent segments simultaneously. This gives a set of
				sorted pieces that then can be \textit{zipped} back together in serial
				for the final sorted array. For sufficiently large arrays, significant
				speed ups can be had here.

				Finally, the interpolation routine is a member function that returns
				a new array, given a new sample grid. By default, the function
				returns strait lines out of the ends of the data, sticking true to the
				\textit{natural} in natural cubic spline interpolation. That is,
				the second derivative of the ends of the outside intervals is by definition
				zero. As a result, any new sample value outside the outermost intervals
				should get evaluated as a strait line off the first derivative of the last
				data point. However, it has as an option to return a null result. This is
				important for the resampling process between trial MCRs. The positions
				of the planets in the model are used as the data to construct new spline
				polynomials. When the fixed radial line-space is used to interpolate on
				those intervals, the outer intervals are very likely to be outside the
				data. Instead of returning poor results, we return zero and track the number
				of times this occurs on a given grid point. In this way, we can keep more
				accurate results only with larger uncertainty.

			\subsection{Framework}

				The essential body of the code is found in the application class
				\texttt{GAIAsolution}. This object is the primary owner of the many
				arms of the code, each responsible for a different set of tasks.
				The application is then wrapped in a main function as an executable.
				In the following sections I'll outline these different components
				and their responsibilities.

			\subsubsection{Main}
				
				The executable is compiled from \texttt{main.cc}. This code is simply
				a wrapper for the application. Although it does perform some tasks.
				The \texttt{monitor} is a singleton, and so it can be grabbed by the
				other objects during runtime. Main instantiates this object first
				because it keeps a set of clocks, one of them is the total simulation
				run time. As such, the very first thing that happens is the initialization
				of the monitor to start the simulation clock. Also, after the application
				object is created and initialized (with the arguments passed to main),
				we check to see if we are actually going to run the code or if we simply
				want to perform the auto-calibration routine. Main decides which of these
				two will happen. Lastly, if the simulation is being run in \textit{verbose}
				mode, main displays the total time elapsed after the simulation is finished
				running. This is simply a top level function the puts everything together.

			\subsubsection{The application class}

				As stated previously, \texttt{GAIAsolution} is the primary application
				object. It owns or operates the majority of the essential arms of the code. 
				Namely, it creates and uses the \texttt{parser}, the \texttt{FileManager}, 
				the \texttt{monitor}, and the \texttt{PopulationManager}. It only has two
				functions, the initialization routine and the actual simulation function.

				The initialization function takes the arguments passed from main and hands
				them off to the \texttt{parser}. After creating and initializing all the
				objects, it verifies that the initialization was a success. All of the objects
				maintain the same design pattern of initialization with a flag that signifies
				a failure. If any daughter object fails in its proper initialization, the parent
				object catches the failure and signals its own failure and tosses it up the line.

				The simulation function is largely just a set of nested loops that iterates over
				all the population sets to be models. Each population is then iterated of the
				requested number of trials. For every trial iteration, a new population (MCR) must
				be built. Then the nearest neighbor analysis must be performed. After we solve
				for the separations, we need to resample this data onto a fixed grid and save it
				for later retrieval. The \texttt{PopulationManager} performs these tasks, but
				the application class delegates them.

			\subsubsection{Parsing input}
				
				When the code is executed, there is a moderate list of default parameters.
				All of these options can be changed, either by passing the argument at the
				command line or by specifying it in one of the configuration files. All of 
				this input needs to be interpreted. The \texttt{parser} first takes the
				arguments that optionally were passed originally from main. It checks to see
				if they are valid and than swaps the default parameter with the new value.

				Next, it reads in and interprets the \textit{parameter file}, followed by
				the \textit{population file}. If the \texttt{--auto-calibrate} flag was
				not given, it signals that we are going to run the code and reads in the
				\textit{calibration file} that was created by the \texttt{calibration}
				object. This object is a \textit{singleton}. In this way, it can parse
				the input data and simply hold onto it, waiting for the other objects to
				request specific pieces of information without the need to pass such
				information as arguments.
				
				These processes are strait forward. The code for this object is
				the longest in the project merely because it contains an extensive list
				of error messages that it gives the user at every possible misstep. There
				are a lot of things that need to be in place for the code to run and the
				parser makes sure they are all in order, signaling the user what the
				particular problem was that halted the simulation before it started.

			\subsubsection{Managing file input/output}
				
				The \texttt{FileManager} is in charge or writing the output files and also
				reading in temporary files created by GAIA. When the simulation is initialized,
				the \texttt{FileManager} asks the \texttt{parser} for information about how
				many population sets will be run and what sizes. It then constructs the output
				file names with appropriate extensions for repeat population sizes. After all
				the trials are complete for a given population size and the analysis is
				finished, the \texttt{FileManager} writes the results to disk as an ascii file.
				The file is in a three column format. The first column is always the radial
				line-space provided in the parameter file. The second and third columns are
				the mean and sample standard deviation respectively. Because the resolution of
				the radial line-space and the number of trials for a given population size would
				have potentially required more active memory to maintain matrices of such size,
				in order to guard the simulation from memory errors, GAIA outputs the results
				for each MCR into a temporary binary file that it understands. After all the
				trials are complete, during the analysis procedure, all of these binary files
				are read from one at a time.

			\subsubsection{Monitoring progress}
				
				As was eluded to in the description of \texttt{main}, the \texttt{monitor}
				is responsible for maintaining a set of clocks. When it is first created,
				it starts the simulation runtime clock. This object also follows the 
				singleton design patter, but this time only so that it doesn't lose the
				time. The primary responsibility for the \texttt{monitor} isn't about
				how long GAIA runs however, it is more important that it monitor progress.
				For any major use of the code, the program would likely be executed with
				the \texttt{--set-verbose=1} option (as oppose the to the default value of 2). 
				This suppresses the following function entirely, because GAIA ought to be
				run on a server for an extended period of time in the background. However,
				for shorter runs, the \texttt{monitor} has a member function that displays
				a progress bar between trials. The progress function is more sophisticated
				than that though. Any run where the user would be actively monitoring the
				progress of the simulation is likely to be of a small population size.
				The iteration time on any population roughly $N\lesssim10^3$ is quite fast.
				Simply putting a print statement inside the loop blindly however often results
				in the simulation being slowed merely because it is trying to print to the
				screen every time. With this in mind, the \texttt{monitor} keeps track of
				the frequency with which it's called upon and suppresses its own output,
				only refreshing the progress statement at a fixed frequency. By default this
				is half of one second.

				In addition, the \texttt{monitor} includes with this progress statement
				the estimated date and time that a given run will be complete based on its
				current performance.

			\subsubsection{Population management}
				
				The real action takes place inside the \texttt{PopulationManager}. It is
				responsible for not only the construction of new MCRs but also the 
				analysis. In order for it to generate new populations it needs a set
				of cumulative distribution functions. For this, it creates and initializes
				the \texttt{CDFmanager}. I'll discuss this process in the sections following
				this one. Once the CDFs are created (for each dimension on which it applies),
				the \texttt{PopulationManager} uses the \texttt{splinter} object to
				construct a set of spline polynomials on the CDF arrays. 

				Every time the build process is requested, the \texttt{PopulationManager}
				generates random number arrays and decides whether to use them to produce
				uniform distributions (on some domain) or if there is a CDF to interpolate.
				Depending on the population size and the information provided from the
				calibration file via the \texttt{parser}, the \texttt{PopulationManager}
				can create and initialize a family of parallel PRNGs to work on
				the arrays.

				The nearest neighbor search is strait forward. As we have to know every
				distance to every other planet, there is no real opportunity for a 
				sophisticated approach. The function that is in charge of this task though
				has two versions of itself. If the population size is small enough, we
				can manage a two dimensional array of size $N^2$ (not the case for most
				instances). If this is in fact the case (and the default threshold is
				$N = 10^4$), we need not double measure. The $r_{ij}$ separation is the
				same as the $r_{ji}$ separation. In this way, we can reduce the total
				computations necessary. Unfortunately, there isn't anything to be done
				for larger population sizes. However, this operation is parallelized 
				using the OpenMP library (as are all of the other parallelizations in GAIA).

				The final analysis routine reads in all the temporary binary files created
				by the \texttt{FileManager} after every iteration and co-adds them for
				a mean value and then computes the standard deviations.


			\subsubsection{CDF management}
				
				While all the other essential components of the code are necessary, the profiles
				that the user provides are the real science. In order for the \texttt{PopulationManager}
				to construct new MCRs it needs CDFs to interpolate onto. When the 
				\texttt{CDFmanager} is created and initialized, it owns a list of \textit{known}
				PDFs. By this I mean it has a list of all that have been defined for it. It asks
				the \texttt{parser} for which PDFs we want to use and it initializes them.
				When a particular PDF is initialized, it is subsequently put onto a separate list specific
				to the dimension (i.e., radial, angular, or vertical). In order to maintain such lists,
				all of the PDF class objects must be derived from the same base class. This is
				\texttt{PDFbase}, described in the next section.

				After picking all the profiles and either evaluating them on the appropriate line-space or
				reading them in from a file and re-sampling, the \texttt{CDFmanager} multiplies all PDFs
				together along a given dimension. Typically, this means that the radial PDFs will be multiplied
				together.

				After we have the master profile in all dimensions, we must integrate. GAIA uses Simpson's
				3/8 rule to solve the general integral. The \texttt{CDFmanager} uses \texttt{splinter} to
				interpolate a set of two midpoints between all the grid points and applies the rule on
				each interval to include a cumulative summation, resulting in the general integral.

				The \texttt{CDFmanager} maintains these CDFs (or whether or not they even exist) and passes
				them off to the \texttt{PopulationManager}.

			\subsubsection{PDF base}
				
				The \texttt{PDFbase} class is an abstract class. It is composed of many \textit{pure virtual}
				functions. It exists to facility the use of lists to maintain the profiles. It describes
				the essential structure of the derived PDFs. Originally, it was thought that this was the
				best way to go, having a nearly pure abstract class. However, it really only needs to be
				the case that function describing the analytical form of the PDF (if desired to have one)
				be virtual (such that it can be redefined). At present, all derived PDF class objects have
				a redefinition of the initialization functions (which are identical). This is not necessary,
				but currently there is no performance loss so it hasn't been changed. In future versions,
				this will be addressed for the sake of clarity and simplicity.

			\subsubsection{Derived PDF classes}
				
				The profile objects are derived  from the \texttt{PDFbase} class. Every new profile needs
				to have a dimension specified at the very least. If your profiles are going to be read in 
				from files, than little differs between the code of these objects apart from the names.
				If your profile is going to take on some analytical form, that must be specified. There
				is not much to say here except point to the following sections where I describe the user
				interface and how to create new profiles.

		\section{User Interface}

		\subsection{Configuration and new profiles}

			The package comes with the following directory structure:

			\begin{Table}
			\begin{tabular}{ll}
				\texttt{LICENSE} & \\
				\texttt{README.md} & \\
				\texttt{data/:} & \\
				& \\
				\texttt{example/:} & \\
				& \texttt{calibration.config} \\
				& \texttt{parameters.config} \\
				& \texttt{population.config} \\
				\texttt{src/:} & \\
				& \texttt{...}\\
				\texttt{tmp/:} & \\
				& 
			\end{tabular}
			\end{Table}

			The \texttt{LICENSE} file contains the GPL v2.0 license. \texttt{README.md} contains
			much of the same information being described here. The directories \texttt{data/}
			and \texttt{tmp/} are empty. They are the default names of the directories to which
			GAIA writes the final results and the temporary binary files respectively. Alternatives
			are discussed in the next section. The \texttt{example/} directory contains three
			configuration files. The names of these files are the default names that GAIA
			looks for if no alternative is specified. They are meant as an illustrative example
			of how to construct these files. The user can move these out of \texttt{example/}
			and alter them for their own purposes.

			The \texttt{src/} directory contains all the source code. One can build GAIA simply
			by executing:

			\begin{center}
				\texttt{cd src/ \&\& ./configure \&\& make}
			\end{center}
			
			This is not particularly helpful however because we haven't made any profiles yet.
			Let me start by saying that the \texttt{configure} script is not yet the conventional
			script users might think it is. It is simply a bash script with the same 
			name. This script compiles three \small{C++} programs and runs two of them.	As was
			described earlier, the \texttt{CDFmanager} is responsible for maintaining a list
			of \textit{known} PDFs. The convention here is that any file that ends in \textit{Profile}
			is expected to be a derived PDF class. There are several pre-made options available
			in an example directory:

			\begin{Table}
			\begin{tabular}{ll}
				\texttt{src/example/:} & \\
				& \texttt{HabitabilityProfile.cc} \\
				& \texttt{HabitabilityProfile.hh} \\
				& \texttt{MassDensityProfile.cc} \\
				& \texttt{MassDensityProfile.hh} \\
				& \texttt{MetallicityProfile.cc} \\
				& \texttt{MetallicityProfile.hh} \\
				& \texttt{VerticalProfile.cc} \\
				& \texttt{VerticalProfile.hh}
			\end{tabular}
			\end{Table}

			The user can move these out of their directory and use them immediately, or modify their
			analytical form by going into their source file. The more general option is to run the
			the configure script once and then use the automated tool provided with GAIA to make
			your profile for you. As an example:

			\begin{Table}
			\begin{tabular}{ll}
				\texttt{~/ProjectGAIA/scr/: ./configure} & \\
				\texttt{~/ProjectGAIA/scr/: make profile} & \\
				\texttt{./makepdf} & \\
				\texttt{Name of new Profile: myNew} & \\
				\texttt{Dimension for myNewProfile ([0],1,2): 1} & \\
				\texttt{Analytical form ([none]): A * sin( omega * t ) } & \\
				\texttt{What was the dependent variable ([r]): t} & \\
				\texttt{List parameters: A, omega} & \\
				\texttt{Value for [A]: 1.0/2.0} & \\
				\texttt{Value for [omega]: 3.1415926 } & \\
				\texttt{./configure} &
			\end{tabular}
			\end{Table}

			This automated tool allows the user to construct new profile objects. Simply
			by answering these questions, the package automatically constructs the new
			source and header file, updates \texttt{CDFmanager.cc} to include it
			in the list of known PDFs, and also updates the \texttt{makefile} to include
			it in the build process. Every time the user creates a new profile, 
			\texttt{make} will need to be executed again to update everything.

		\subsection{Input files}

			When GAIA runs, it looks for three input files. By default, if unspecified,
			it expects \texttt{parameters.config}, \texttt{population.config}, and
			\texttt{calibration.config} to exist in the current directory (with an
			exception on the last one).

			The parameters file is where you tell GAIA what your line-spaces are and
			what profiles you want to include out of the known profiles in this run.
			You can write C-style comments with double slashes and leave any amount
			of blank lines. Furthermore, it doesn't matter what order you include
			your input. But lines must follow a specific format. First, line-spaces
			must be one of three: \texttt{radial\_linespace}, \texttt{vertical\_linespace},
			or \texttt{angular\_points}. For the angular line-space it is assumed to range
			over $0 - 2\pi$. For radial and vertical the format is as follows:

			\begin{center}
				\texttt{ <type\_linespace> : <start> <end> <num\_points> }
			\end{center}

			So to model a disk out to 15 kpc\footnote{A note about units: 
			it is up to the user to be consistent with units. The units 
			used for the radial and vertical line-space should be the same
			and they will be the units the results are reported in.}
			and have a half-parsec resolution:

			\begin{center}
				\texttt{ radial\_linespace : 0 15000 30001 }
			\end{center}
			
			The number of angular points is a single number not three. The value for the 
			number of angular points as well as in place of the three numbers for the
			vertical line-space, the special keyword, \texttt{none}, can be substituted.
			In the case of the angular points it means we are isotropic and will
			distribute angles uniformly. In the case of the vertical line-space it means
			we have no vertical structure and only a flat disk. All three of these
			need to be provide and the radial line-space is demanded.

			The profiles are specified with the keyword, \texttt{profile}, and their
			inclusion takes the following pattern:

			\begin{center}
				\texttt{ profile : <name of profile> <value> }
			\end{center}

			\noindent
			where \texttt{<value>} should be either the path to the file for initializing
			the PDF or the special keyword, \texttt{linespace}, which means we aren't
			going to read that one from a file but evaluate it using it's analytical form
			along the line-space that corresponds to its dimension. For an example, see
			\texttt{ProjectGAIA/example/parameters.config}.
			
			The population file is much simpler. It should be an ascii file containing two
			columns of numbers. Every row represents a different population set, where
			the first number is the size of the population and the second number is the
			quantity of trials to build. As an example:

			\begin{Table}
			\begin{tabular}{ll}
				\texttt{512} & \texttt{1000} \\
				\texttt{16384} & \texttt{128} 
			\end{tabular}
			\end{Table}

			\noindent
			says to build two population sets. We are to build 1000 MCRs of size 512 and then
			128 MCRs of size 16384.

			Finally, \texttt{calibration.config} should take a similar format to the 
			population file, but have a set of 16 numbers for thread counts instead of a number
			for the trials. For example:

			\begin{Table}
			\begin{tabular}{ll}
				\texttt{512} & \texttt{1 1 1 1 1 1 1 1 1 1 1 1 1 1 1 1} \\
				\texttt{16384} & \texttt{1 1 1 1 1 1 1 1 1 1 4 1 1 1 1 1}
			\end{tabular}
			\end{Table}

			\noindent
			says for the same population sizes, we want to run everything in serial for sizes
			512, but for the second population, 16384, we want to spawn four threads for the
			11th task (which happens to be the nearest neighbor search). For \textit{this}
			configuration file, GAIA has a calibration routine that will auto generate this
			file based on empirical measurements and the contents of your population file
			and the maximum number of threads the user stipulates it's allowed to use.
			This is at least how it will be for the next version for the code. The 
			\texttt{calibration} object has not yet been implemented. Furthermore, in future
			editions of the code the \texttt{greeter} will be implemented which runs GAIA
			in an optional interactive mode and requests information from the user, only
			to then actually auto generate the population and parameter files.

		\subsection{Execution}

			With everything assumed to be default and the configuration files present and
			accounted for, one simple must execute: \texttt{./gaia.app}. The following in
			Table \ref{Table:flags} is a list of the optional flags that can be provided 
			at run time:

			\begin{Table}
			\captionof{table}{Table of GAIA flags.}
			\begin{tabular}{rcl} \hline\hline
				& & \\
				\texttt{--set-verbose} & \texttt{=} & \texttt{0 - silent} \\
				& & \texttt{1 - some output} \\
				& & \texttt{2 - progress bar} \\
				\texttt{--output-directory} & \texttt{=} & \texttt{/path/to} \\
				\texttt{--temp-directory}   & \texttt{=} & \texttt{...}\\
				\texttt{--parameter-file}   & \texttt{=} & \texttt{...}\\
				\texttt{--population-file}  & \texttt{=} & \texttt{...}\\
				\texttt{--calibration-file} & \texttt{=} & \texttt{...}\\
				\texttt{--interactive-mode} & & \\
				\texttt{--auto-calibrate}   & & \\
				\texttt{--max-threads}      & \texttt{=} & \texttt{integer $>$ 0} \\
				\texttt{--force-threads}    & & \texttt{...}\\
				\texttt{--size-threshold}   & \texttt{=} & \texttt{integer $>$ 3}\\
				\texttt{--trial-sets}       & \texttt{=} & \texttt{integer $>$ 0}
			\end{tabular}
			\label{Table:flags}
			\end{Table}
			
			The parameters do what they sound like. All the arguments must be
			passed with an equal sign followed by the value with no spaces
			with the exception of the interactive mode and auto calibration
			flag with should not be given with an equal sign. The size threshold
			argument is the cutoff point for the auto calibration routine to 
			stop testing parallelization. By default it will go up to the largest
			size in the population file or until it reaches the maximum number
			of threads or until it reaches the maximum size for its data type,
			which ever is first. The trial sets argument says how many times
			to test the tasks during the auto calibration routine. By default it
			only tests it once.

		\section{Discussion}
		
			I have as of yet not had the opportunity to make a real run with
			the program, so I won't be including plots of output.
			I need to do more research on the functional form I might use for
			the profiles I suggested. Thus far I've tested the algorithm
			with merely the radial profile without vertical structure. The
			results show something reminiscent of a root function, with increasing
			noise with distance from the center of the galaxy. I expect with
			larger population sizes, not only for the separations to decrease
			(as they must)
			but the noise to diminish to some extent. I think with the inclusion
			of a habitability profile, there will be a dip in the function near
			the orbit of co-rotation. The behavior of this curve is not
			necessarily surprising in any way, but the actual value of the function
			at Earth's radius and the standard deviation in the expected separations
			is what is of interest.
			It won't be until I can provide valid input that I get valid output.

			This code is a tool that I hope to continue to develop in the hopes
			that it is useful to me to do real meaningful work in the SETI field.
			As I stated earlier, it needs a more sophisticated treatment of the 
			probability and statistics. Also, it could be useful in its own
			right to have it optionally output the actually positions of the
			bodies for a single trial as a tool for generating initial positions
			for galaxy N-body codes.

			As for the experience of the project, it has provided many
			challenges for me to overcome and I think I've succeeded in most of them. 
			I spent a month getting the spline interpolation working correctly
			with the quicksort algorithm. Typically, when I ran into a problem
			getting a design or logic to work the way I wanted, I would research
			how others had done it, then implement my own version of that 
			technique and get it to work from there.

		\section{Acknowledgment}
			
			In addition to thanking Professor Brown for his patience and continuous
			help on this project, I'd like to acknowledge my colleagues in the
			Astrophysics lab for their 
			numerous opinions and constructive feedback, particularly Brian Leist
			for encouraging me to pursue it in the first place.

			As well, I think it's appropriate to acknowledge Dr. Duncan Forgan
			\footnote{University of St. Andrews}
			with whom I've only had brief correspondence but whose papers were my
			only insight into what has been attempted regarding Monte Carlo methods
			applied to SETI in this way. I failed to cite his work during the 
			brief introductory sections of this paper \cite{forgan1, forgan2}.
			I hope to get feedback from him and others working in this area.

		\section*{References}
		\bibliographystyle{ieeetr}
		\bibliography{ProjectGAIA_writeup}

	\end{multicols}

	%	\pagebreak

	\section{Apendix: Source Code}
	\subsection{makefile}
	\VerbatimInput[baselinestretch=1,fontsize=\footnotesize, numbers=left]{../source/makefile}

	\pagebreak
	
	\subsection{main.cc}
	\VerbatimInput[baselinestretch=1,fontsize=\footnotesize, numbers=left]{../source/main.cc}

	\pagebreak
	
	\subsection{GAIAsolution.hh}
	\VerbatimInput[baselinestretch=1,fontsize=\footnotesize, numbers=left]{../source/GAIAsolution.hh}

	\pagebreak

	\subsection{GAIAsolution.cc}
	\VerbatimInput[baselinestretch=1,fontsize=\footnotesize, numbers=left]{../source/GAIAsolution.cc}

	\pagebreak

	\subsection{parser.hh}
	\VerbatimInput[baselinestretch=1,fontsize=\footnotesize, numbers=left]{../source/parser.hh}

	\pagebreak

	\subsection{parser.cc}
	\VerbatimInput[baselinestretch=1,fontsize=\footnotesize, numbers=left]{../source/parser.cc}

	\pagebreak

	\subsection{FileManager.hh}
	\VerbatimInput[baselinestretch=1,fontsize=\footnotesize, numbers=left]{../source/FileManager.hh}

	\pagebreak

	\subsection{FileManager.cc}
	\VerbatimInput[baselinestretch=1,fontsize=\footnotesize, numbers=left]{../source/FileManager.cc}

	\pagebreak

	\subsection{monitor.hh}
	\VerbatimInput[baselinestretch=1,fontsize=\footnotesize, numbers=left]{../source/monitor.hh}

	\pagebreak

	\subsection{monitor.cc}
	\VerbatimInput[baselinestretch=1,fontsize=\footnotesize, numbers=left]{../source/monitor.cc}

	\pagebreak

	\subsection{PopulationManager.hh}
	\VerbatimInput[baselinestretch=1,fontsize=\footnotesize, numbers=left]{../source/PopulationManager.hh}

	\pagebreak

	\subsection{PopulationManager.cc}
	\VerbatimInput[baselinestretch=1,fontsize=\footnotesize, numbers=left]{../source/PopulationManager.cc}
	
	\pagebreak

	\subsection{CDFmanager.hh}
	\VerbatimInput[baselinestretch=1,fontsize=\footnotesize, numbers=left]{../source/CDFmanager.hh}

	\pagebreak

	\subsection{CDFmanager.cc}
	\VerbatimInput[baselinestretch=1,fontsize=\footnotesize, numbers=left]{../source/CDFmanager.cc}

	\pagebreak

	\subsection{PDFbase.hh}
	\VerbatimInput[baselinestretch=1,fontsize=\footnotesize, numbers=left]{../source/PDFbase.hh}

	\pagebreak

	\subsection{MassDensityProfile.hh}
	\VerbatimInput[baselinestretch=1,fontsize=\footnotesize, numbers=left]{../source/MassDensityProfile.hh}

	\pagebreak

	\subsection{MassDensityProfile.cc}
	\VerbatimInput[baselinestretch=1,fontsize=\footnotesize, numbers=left]{../source/MassDensityProfile.cc}

	\pagebreak

	\subsection{MetallicityProfile.hh}
	\VerbatimInput[baselinestretch=1,fontsize=\footnotesize, numbers=left]{../source/MetallicityProfile.hh}

	\pagebreak

	\subsection{MetallicityProfile.cc}
	\VerbatimInput[baselinestretch=1,fontsize=\footnotesize, numbers=left]{../source/MetallicityProfile.cc}

	\pagebreak

	\subsection{HabitabilityProfile.hh}
	\VerbatimInput[baselinestretch=1,fontsize=\footnotesize, numbers=left]{../source/HabitabilityProfile.hh}

	\pagebreak

	\subsection{HabitabilityProfile.cc}
	\VerbatimInput[baselinestretch=1,fontsize=\footnotesize, numbers=left]{../source/HabitabilityProfile.cc}

	\pagebreak

	\subsection{VerticalProfile.hh}
	\VerbatimInput[baselinestretch=1,fontsize=\footnotesize, numbers=left]{../source/VerticalProfile.hh}

	\pagebreak

	\subsection{VerticalProfile.cc}
	\VerbatimInput[baselinestretch=1,fontsize=\footnotesize, numbers=left]{../source/VerticalProfile.cc}

	\pagebreak

	\subsection{TemplateProfile.hh}
	\VerbatimInput[baselinestretch=1,fontsize=\footnotesize, numbers=left]{../source/TemplateProfile.hh}

	\pagebreak

	\subsection{TemplateProfile.cc}
	\VerbatimInput[baselinestretch=1,fontsize=\footnotesize, numbers=left]{../source/TemplateProfile.cc}


	
	
\end{document}	
