\documentclass[10pt,a4paper,final]{article}
\usepackage[utf8]{inputenc}
\usepackage[english]{babel}
\usepackage{amsmath}
\usepackage{amsfonts}\usepackage{amssymb}
\usepackage{makeidx}
\usepackage{graphicx}
\usepackage{lmodern}
\usepackage[left=1in,right=1in,top=1in,bottom=1in, headheight=15pt]{geometry}
\usepackage[bottom]{footmisc}
\usepackage{nopageno}
\usepackage{fancyhdr}
\pagestyle{fancy}
\fancyhf{}
\fancyhead[L]{\small{Physics 565}}
\fancyhead[C]{\small{Computational Physics}}
\fancyhead[R]{\small{Term Project - Proposal}}


\begin{document}

	\begin{flushleft}
		\par{\Large{\textbf{Numerical Methods in Quantifying Estimates on
							the Number of Habitable Bodies in the Milky Way Galaxy}}\par}
	\end{flushleft}

	\bigskip

	\begin{center}
		\par{28 October 2014\par}
		\bigskip
		\par{Geoffrey Lentner
		\footnote{Department of Physics \& Astronomy, University of Louisville}
		\par}
	\end{center}

	\vspace*{\baselineskip}

	\section{Overview}

		In the wake of the Kepler mission, there has been a boom in the discovery
		and characterization of exoplanets, and along with that a revival in optimism
		regarding the existance of other habitable worlds in our Galaxy. One significant
		question of interest is whether we exist in a relatively prefered location or
		not in terms of the distance to the nearest habitable planet. Where are
		we most likely to find our nearest Earth-like neighbor; what is the confidence
		in this measure.
		
		The principle target of investigation for this project is to estimate
		the expectation and variation in the seperations between habitable planets
		as a function of galactocentric radius. That is, if we assume a probability
		distribution function for the galaxy and a total number $N$, we can distribute
		these $N$ bodies around the galaxy stochastically. This is a Monte Carlo
		Realization (MCR) of these planets in the Milky Way. For many MCRs, we can
		solve for the distance to the nearest neighbor of each body, then catalog
		by radius, to build a function $S(R,N)$ for the expected seperation and
		for the standard deviation in this measure.

	
	\section{Implementation}

		I am developing an application for this purpose, implemented in C++, which builds
		MCRs of the galaxy. The validity of the output is dependent on the input. I've
		included a number of probability distribution functions (PDFs) that are ``known''
		to the program, but this will be easily alterable and extensible. The PDFs
		will be combined and integrated to build a cummulative distribution function (CDF)
		such that a random variable may be interpolated to give position vectors. A subroutine
		that manages this process, will have a generating function for building these
		populations. The main program will iterate over this build process for some desired
		set of values for $N$ and a given number of trials for each value. Once the
		populations have been constructed, the nearest neighbor will be found for every
		body and the results sorted by radius. In order to combine each subsequent trial,
		the results need to be resampled onto a fixed grid via interpolation. The
		results will be analyzed for sample mean and standard deviation. The output
		will be saved to text files, one for every value $N$.
	
	\section{Analysis}	
	
		I expect that in general, as one moves further out into the disk, the expected
		distance to the nearest neighbor will grow. I also expect that as the number
		of habitable planets in the galaxy increases, the standard deviation will
		diminish. My hypothesis is that I will be able to constrain the number of
		possible habitable planets in the galaxy considering knowledge of our own
		environment. If we can establish within a certain confidence interval that there
		are no habitable planets within a given volume, with that same confidence
		we can constrain the total allowed planets (understanding that this result
		is entirely dependent on our choice of PDFs).
		
\end{document}	
